\documentclass{beamer}

\usepackage{default}

\usepackage[utf8]{inputenc}

\begin{document}

\begin{frame}[fragile=singleslide]\frametitle{Quiz 6}

\begin{block}{Ejercicio único}
\begin{small}
A continuación se presenta el código de una librería básica. El archivo presentado tiene el nombre \verb|Perro.cpp|. Presente el código del archivo \verb|Perro.h|. 
\end{small}
\end{block}

\vspace{-0.3in}
\begin{columns}[t]
\column{.45\textwidth}

\begin{scriptsize}
\begin{verbatim}
#include "Perro.h"
#include <Arduino.h>

Perro::Perro(int p, int q) {
  pinMode(p, OUTPUT);
  pinMode(q, INPUT);
  pin = p;
  control = q;
}

Perro::ladrar() {
  if (digitalRead(control)==HIGH) {
  	analogWrite(pin, 128);
  } else {
  	analogWrite(pin, 255);	
  }
}
\end{verbatim}
\end{scriptsize}

\column{.45\textwidth}
\begin{scriptsize}
\begin{verbatim}
Perro::alto() {
	analogWrite(pin, 0);
}

Perro::correr() {
	digitalWrite(pin, HIGH);
}

Perro::comer() {
  if (digitalRead(control)==LOW) {
  	digitalWrite(pin, LOW);
  }
}
\end{verbatim}
\end{scriptsize}

\end{columns}
\end{frame}

\end{document}
