\documentclass[12pt,letterpaper]{IEEEtran}
\usepackage[utf8x]{inputenc}
\usepackage[spanish]{babel}
\usepackage{enumitem}

\title{Laboratorio 6: Arduino -- Librerías}
\author{Prof. Ing. Jorge Rivera G.}
\date{\today}

\newcommand\MYhyperrefoptions{bookmarks=true,bookmarksnumbered=true,
pdfpagemode={UseOutlines},plainpages=false,pdfpagelabels=true,
colorlinks=true,linkcolor={black},citecolor={black},
urlcolor={black}}

\usepackage[\MYhyperrefoptions]{hyperref}

\begin{document}
\hypersetup{pdftitle={Laboratorio 6: Arduino -- Librerías},
pdfsubject={IEC-650 Laboratorio de Sistemas Digitales},
pdfauthor={Ing. Jorge Rivera},
pdfkeywords={arduino, sistemas digitales}}

\renewcommand{\leftmark}{UNIVERSIDAD LATINA DE COSTA RICA -- IEC-650 LABORATORIO DE SISTEMAS DIGITALES}

\maketitle


\begin{abstract}
Esta es una práctica para aprender el diseño y desarrollo de librerías para la plataforma Arduino.
\end{abstract}

\section{Descripción}

Esta práctica será realizada enteramente en el laboratorio. Los estudiantes se organizarán en parejas. Cada pareja realizará un pequeño circuito utilizando una tarjeta Arduino.

\section{Materiales}

Para esta práctica se necesitarán los siguientes materiales.

\begin{center}
\begin{tabular}{c|c}\hline
	Cant. & \hspace{2cm}Material\hspace{2cm} \\\hline\hline
	1 	& Arduino o tarjeta similar		\\\hline
	1	& Computadora con puerto USB  	\\\hline	
\end{tabular}
\end{center}

\section{Requerimientos}

Para la conclusión satisfactoria de este laboratorio se deberán cumplir con los siguientes requerimientos:

\begin{itemize}
	\item Crear una clase que represente a un LED.
	\begin{itemize}
	\item Crear una función que encienda el LED.
		\item Crear una función que apague el LED.
		\item Crear una función que haga que el LED parpadeé 3 veces rápidamente.
		\item Crear una función que haga que el LED parpadeé 3 veces lentamente.
		\item Crear una función que haga que el LED brille progresivamente usando PWM.		
	\end{itemize}
	\item Crear dos instancias de la clase. 
	\item Crear un menú sencillo usando el puerto serial para utilizar las funciones de la librería.
\end{itemize}

\section{Procedimiento}

\subsection{Charla preliminar}

\begin{enumerate}
	\item El profesor impartirá una breve charla sobre conceptos de importancia en el desarrollo del proyecto. Se deberá poner atención a esta charla y participar de forma activa con las preguntas y dudas que surjan.
\end{enumerate}


\subsection{Programación}

\begin{enumerate}[resume]
	\item Se deberá escribir el programa en la interfaz de Arduino.
\end{enumerate}

\subsection{Verificación}

\begin{enumerate}[resume]
	\item Cuando se haya logrado completar uno o más requerimientos, el funcionamiento parcial deberá ser mostrado al profesor.
	\item Cuando se haya concluido con todos los requerimientos, se deberá realizar una última demostración al profesor y se concluirá la práctica.
\end{enumerate}


\section{Informe}

El informe que se deberá presentar constará de las siguientes partes:

\begin{enumerate}
  \item Encabezado
  \item Resumen o abstract
  \item Descripción del circuito
  \item Listado de materiales
  \item Esquemático o diagrama del circuito
  \item Listado del programa realizado con comentarios
  \item Conceptos aprendidos
\end{enumerate}

El informe deberá realizarse utilizando el sistema de preparación de documentos \LaTeX, utilizando el formato IEEEtran. El documento deberá ser entregado en forma impresa en la clase correspondiente y en forma digital en el Aula Virtual, incluyendo el código fuente y el resultado en PDF.  La fecha de entrega será una semana natural después de la realización de la práctica.

La lista de componentes deberá incluir los circuitos integrados y elementos activos utilizados en la práctica con sus números de parte detallados, la cantidad y valor de los elementos pasivos utilizados y cualquier otro elemento eléctrico utilizado en el circuito. No se deberán incluir cables, bases para circuitos integrados, protoboards, etc.

Para realizar el esquemático del circuito, se deberá utilizar una herramienta apropiada. Utilice símbolos apropiados para un esquemático. No son admisibles diagramas realizados en Microsoft Paint, Adobe Photoshop u otras herramientas similares. Se sugiere utilizar Fritzing.

Para presentar el listado del programa se deberá utilizar la fuente \texttt{typewriter} de \LaTeX\ o el ambiente \texttt{verbatim}, y con las tabulaciones correctas. Se deberán presentar los comentarios adecuados para el código realizado en la línea inmediatamente anterior al código que se esté comentando. No se deberán mezclar comentarios y código en la misma línea.

Los conceptos aprendidos deberán ser una lista de notas importantes que se hayan recogido durante la clase. 

Se castigará duramente el intento de plagio.

\section{Evaluación}

La evaluación de este práctica será con una calificación de 0 a 100, distribuida de la siguiente forma:

\begin{center}
 \begin{tabular}{p{0.35\textwidth}|c}\hline
   Funcionamiento del circuito (de acuerdo a los requisitos indicados anteriormente) 					     & 30\% \\\hline
   Listado del programa					& 15\% \\\hline
   Conceptos Aprendidos					& 25\% \\\hline
   Encabezado, componentes, descripción & 15\% \\\hline
   Calidad del informe					& 15\% \\\hline\hline
   Total								& 100\% \\
 \end{tabular}
\end{center}

La calificación del rubro de funcionamiento del circuito es condicional a la entrega del informe correspondiente.

De la sección de calidad del informe se sustraerán puntos en caso de que el informe tenga faltas ortográficas, gramaticales u otros errores de forma y presentación.

Si por motivos justificados se requiere de más tiempo para completar el informe, esta deberá ser solicitada al profesor al menos con 24 horas de anticipación.

En caso de no haber solicitado una extensión por anticipado o de haberse vencido la extensión, la máxima nota estará dada por la fórmula:

\[ M(n) = 100-\frac{1.367}{10}\cdot n^{1.367} \]

donde $M$ es la nota máxima y $n$ es la cantidad de horas de atraso en la entrega. La nota final será:

\[ F(n) = M(n)\cdot T \]

donde $T$ corresponde al porcentaje obtenido de los rubros especificados anteriormente.

\end{document}