\documentclass[12pt,letterpaper]{IEEEtran}
\usepackage[utf8x]{inputenc}
\usepackage[spanish]{babel}
\usepackage{enumitem}

\title{Laboratorio 2: Arduino - puerto serial}
\author{Prof. Ing. Jorge Rivera G.}
\date{\today}

\newcommand\MYhyperrefoptions{bookmarks=true,bookmarksnumbered=true,
pdfpagemode={UseOutlines},plainpages=false,pdfpagelabels=true,
colorlinks=true,linkcolor={black},citecolor={black},
urlcolor={black}}

\usepackage[\MYhyperrefoptions]{hyperref}

\begin{document}
\hypersetup{pdftitle={Laboratorio 2: Puerto Serial de Arduino },
pdfsubject={IEC-650 Laboratorio de Sistemas Digitales},
pdfauthor={Ing. Jorge Rivera},
pdfkeywords={arduino, sistemas digitales}}

\renewcommand{\leftmark}{UNIVERSIDAD LATINA DE COSTA RICA -- IEC-650 LABORATORIO DE SISTEMAS DIGITALES}

\maketitle


\begin{abstract}
Esta es una práctica para aprender el funcionamiento de la consola serial y las estructuras switch en el ambiente Arduino. Se utilizarán LEDs controlados por el usuario desde la computadora.
\end{abstract}

\section{Descripción}

Esta práctica será realizada enteramente en el laboratorio. Los estudiantes se organizarán en parejas. Cada pareja realizará un pequeño circuito con un interruptor y leds, utilizando una tarjeta Arduino.

\section{Materiales}

Para esta práctica se necesitarán los siguientes materiales.

\begin{center}
\begin{tabular}{c|c}\hline
	Cant. & \hspace{2cm}Material\hspace{2cm} \\\hline\hline
	1 	& Arduino o tarjeta similar		\\\hline
	1	& Protoboard 	\\\hline	
	5	& leds			\\\hline
	1	& Interruptor	\\\hline
	---	& Resistencias variadas \\\hline
	--- & Cables		\\\hline
\end{tabular}
\end{center}

\section{Requerimientos}

Para la conclusión satisfactoria de este laboratorio se deberán cumplir con los siguientes requerimientos:

\begin{itemize}
	\item Definir un protocolo de comunicación, con las acciones a seguir para cada dato recibido.
	\item Mostrar la lista de opciones o menú en el puerto serie.
	\item Recibir un dato en el puerto serie.
	\item Procesar el dato recibido y cambiar el estado de los LEDs de acuerdo con lo especificado por el protocolo.
	\item Enviar una respuesta por el puerto serie.
	\item Integrar el funcionamiento de un interruptor que permita retornar los LEDs a un estado inicial.
\end{itemize}

\section{Procedimiento}

\subsection{Charla preliminar}

\begin{enumerate}
	\item El profesor impartirá una breve charla sobre conceptos de importancia en el desarrollo del proyecto. Se deberá poner atención a esta charla y participar de forma activa con las preguntas y dudas que surjan.
\end{enumerate}


\subsection{Preparación}

\begin{enumerate}[resume]
	\item Se deberá realizar el diseño de un circuito, indicando los números de pines a utilizar en la tarjeta y la forma de conexión del resto de los componentes. Este diseño se deberá realizar en papel y ser mostrado al profesor antes de ser alambrado.

	\item Se deberá realizar el diseño del programa en un diagrama de flujo sencillo. Este diagrama de flujo se deberá realizar en papel y ser mostrado al profesor antes de ser programado. Se sugiere utilizar bloques con operaciones sencillas en el diseño de tal forma que se puedan realizar implementaciones parciales.
\end{enumerate}

\subsection{Alambrado del circuito}

\begin{enumerate}[resume]
	\item El circuito se deberá alambrar usando protoboards u otros dispositivos de prototipo apropiados. El alambrado deberá mantenerse ordenado en la medida de lo posible.
\end{enumerate}

\subsection{Programación}

\begin{enumerate}[resume]
	\item Se deberá escribir el programa en la interfaz de Arduino.
\end{enumerate}

\subsection{Verificación}

\begin{enumerate}[resume]
	\item Cuando se haya logrado completar uno o más requerimientos, el funcionamiento parcial deberá ser mostrado al profesor.
	\item Cuando se haya concluido con todos los requerimientos, se deberá realizar una última demostración al profesor y se concluirá la práctica.
\end{enumerate}

\section{Informe}

El informe que se deberá presentar constará de las siguientes partes:

\begin{enumerate}
  \item Encabezado
  \item Resumen o abstract
  \item Descripción del circuito
  \item Listado de materiales
  \item Esquemático o diagrama del circuito
  \item Descripción del protocolo establecido
  \item Descripción del programa realizado
  \item Conceptos aprendidos
\end{enumerate}

El informe deberá realizarse utilizando el sistema de preparación de documentos \LaTeX, utilizando el formato IEEEtran. El documento deberá ser entregado en forma impresa en la clase correspondiente y en forma digital en el Aula Virtual, incluyendo el código fuente y el resultado en PDF.  La fecha de entrega será una semana natural después de la realización de la práctica.

La lista de componentes deberá incluir los circuitos integrados y elementos activos utilizados en la práctica con sus números de parte detallados, la cantidad y valor de los elementos pasivos utilizados y cualquier otro elemento eléctrico utilizado en el circuito. No se deberán incluir cables, bases para circuitos integrados, protoboards, etc.

Para realizar el esquemático del circuito, se deberá utilizar una herramienta apropiada. Utilice símbolos apropiados para un esquemático. No son admisibles diagramas realizados en Microsoft Paint, Adobe Photoshop u otras herramientas similares. Se sugiere utilizar Fritzing.

Para presentar la descripción del programa, se deberá hacer una explicación de cómo funciona el programa. Se podrá ilustrar esta sección con recortes del programa. Los recortes del listado del programa se deberán presentar utilizando el ambiente \texttt{verbatim}, y con las tabulaciones correctas. 

Los conceptos aprendidos deberán ser una lista de notas importantes que se hayan recogido durante la clase. 

Se castigará duramente el intento de plagio.

\section{Evaluación}

La evaluación de este práctica será con una calificación de 0 a 100, distribuida de la siguiente forma:

\begin{center}
 \begin{tabular}{p{0.35\textwidth}|c}\hline
   Funcionamiento del circuito (de acuerdo a los requisitos indicados anteriormente) 					     & 30\% \\\hline
   Planteamiento del protocolo			& 15\% \\\hline
   Descripción del programa				& 15\% \\\hline
   Conceptos Aprendidos					& 20\% \\\hline
   Calidad del informe					& 10\% \\\hline
   Encabezado, componentes, descripción & 10\% \\\hline\hline
   Total								& 100\% \\
 \end{tabular}
\end{center}
La calificación del rubro de funcionamiento del circuito es condicional a la entrega del informe correspondiente.

De la sección de calidad del informe se sustraerán puntos en caso de que el informe tenga faltas ortográficas, gramaticales u otros errores de forma y presentación.

Si por motivos justificados se requiere de más tiempo para completar el informe, esta deberá ser solicitada al profesor al menos con 24 horas de anticipación.

En caso de no haber solicitado una extensión por anticipado o de haberse vencido la extensión, la máxima nota estará dada por la fórmula:

\[ M(n) = 100-\frac{1.367}{10}\cdot n^{1.367} \]

donde $M$ es la nota máxima y $n$ es la cantidad de horas de atraso en la entrega. La nota final será:

\[ F(n) = M(n)\cdot T \]

donde $T$ corresponde al porcentaje obtenido de los rubros especificados anteriormente.


\end{document}