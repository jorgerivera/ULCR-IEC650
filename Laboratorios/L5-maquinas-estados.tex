\documentclass[12pt,letterpaper]{IEEEtran}
\usepackage[utf8x]{inputenc}
\usepackage[spanish]{babel}
\usepackage{enumitem}

\title{Laboratorio 5: M\'aquinas de estados}
\author{Prof. Ing. Jorge Rivera G.}
\date{\today}

\newcommand\MYhyperrefoptions{bookmarks=true,bookmarksnumbered=true,
pdfpagemode={UseOutlines},plainpages=false,pdfpagelabels=true,
colorlinks=true,linkcolor={black},citecolor={black},
urlcolor={black}}

\usepackage[\MYhyperrefoptions]{hyperref}

\begin{document}
\hypersetup{pdftitle={Laboratorio 5: Máquinas de estados},
pdfsubject={IEC-650 Laboratorio de Sistemas Digitales},
pdfauthor={Ing. Jorge Rivera},
pdfkeywords={arduino, sistemas digitales}}

\renewcommand{\leftmark}{UNIVERSIDAD LATINA DE COSTA RICA -- IEC-650 LABORATORIO DE SISTEMAS DIGITALES}

\maketitle


\begin{abstract}
Esta es una práctica donde se utilizarán máquinas de estados para realizar tareas en paralelo sin realizar bloqueos.
\end{abstract}

\section{Descripción}

Esta práctica será realizada enteramente en el laboratorio. Los estudiantes se organizarán en parejas. Cada pareja realizará un pequeño circuito utilizando una tarjeta Arduino.

\section{Materiales}

Para esta práctica se necesitarán los siguientes materiales.

\begin{center}
\begin{tabular}{c|c}\hline
	Cant. & \hspace{2cm}Material\hspace{2cm} \\\hline\hline
	1 	& Arduino o tarjeta similar		\\\hline
	1	& Computadora con puerto USB  	\\\hline	
\end{tabular}
\end{center}

\section{Requerimientos}

Para la conclusión satisfactoria de este laboratorio se deberán cumplir con los siguientes requerimientos:

\begin{itemize}
	\item Controlar el parpadeo de un led con una máquina de estados. El tiempo que el led pasa encendido será igual al de apagado, pero podrá variarse. Inicialmente deberá tener un periodo de 1 Hz.
	\item Controlar el parpadeo de un segundo led utilizando una máquina de estados similar a la anterior. Inicialmente deberá tener un periodo de 2 Hz.
	

	\item Crear una máquina de estados que permita al usuario navegar por una serie de menús. Los menús serán mostrados al usuario en la consola serial. El usuario ingresará su selección usando la consola serial.
	\item Incluir elementos en el menú que permitan realizar incrementos o decrementos relativos al período del LED. Por ejemplo, sumarle o restarle 500ms al tiempo actual.
	\item Incluir elementos en el menú que permitan realizar modificaciones directas al periodo del LED. Por ejemplo, seleccionar 1s o seleccionar 3s.
	\item Incluir elementos en el menú que permitan reestablecer el valor inicial de uno o ambos tiempos.
	\item Incluir elementos en el menú principal que permita imprimr el valor actual de los intervalos de encendido o apagado.
\end{itemize}

\section{Procedimiento}

\subsection{Charla preliminar}

\begin{enumerate}
	\item El profesor impartirá una breve charla de repaso sobre los conceptos de funciones y máquinas de estados.
\end{enumerate}


\subsection{Preparación}

\begin{enumerate}[resume]
	\item Se deberá realizar el diseño del menú a utilizar en papel. Se deberá mostrar la propuesta al profesor antes de ser implementada.
\end{enumerate}


\subsection{Programación}

\begin{enumerate}[resume]
	\item Se deberá escribir el programa en la interfaz de Arduino.
\end{enumerate}

\subsection{Verificación}

\begin{enumerate}[resume]
	\item Cuando se haya logrado completar uno o más requerimientos, el funcionamiento parcial deberá ser mostrado al profesor.
	\item Cuando se haya concluido con todos los requerimientos, se deberá realizar una última demostración al profesor y se concluirá la práctica.
\end{enumerate}


\section{Informe}

El informe que se deberá presentar constará de las siguientes partes:

\begin{enumerate}
  \item Encabezado
  \item Resumen o abstract
  \item Descripción del circuito y el menú
  \item Listado de materiales
  \item Descripción del programa realizado
  \item Conceptos aprendidos
\end{enumerate}

El informe deberá realizarse utilizando el sistema de preparación de documentos \LaTeX, utilizando el formato IEEEtran. El documento deberá ser entregado en forma impresa en la clase correspondiente y en forma digital en el Aula Virtual, incluyendo el código fuente y el resultado en PDF.  La fecha de entrega será una semana natural después de la realización de la práctica.

La lista de componentes deberá incluir los circuitos integrados y elementos activos utilizados en la práctica con sus números de parte detallados, la cantidad y valor de los elementos pasivos utilizados y cualquier otro elemento eléctrico utilizado en el circuito. No se deberán incluir cables, bases para circuitos integrados, protoboards, etc.

Para presentar la descripción del programa, se deberá hacer una explicación de cómo funciona el programa. Se podrá ilustrar esta sección con recortes del programa. Los recortes del listado del programa se deberán presentar utilizando el ambiente \texttt{verbatim}, y con las tabulaciones correctas. 

Los conceptos aprendidos deberán ser una lista de notas importantes que se hayan recogido durante la clase y de los conceptos que se aplicaron en la práctica. Deberá realizarse una breve explicación de cada elemento en la lista.

Se castigará duramente el intento de plagio.

\section{Evaluación}

La evaluación de este práctica será con una calificación de 0 a 100, distribuida de la siguiente forma:

\begin{center}
 \begin{tabular}{p{0.35\textwidth}|c}\hline
   Funcionamiento del circuito (de acuerdo a los requisitos indicados anteriormente) 					     & 40\% \\\hline
   Calidad del informe	  				& 10\% \\\hline
   Descripción del programa				& 20\% \\\hline
   Conceptos Aprendidos					& 20\% \\\hline
   Encabezado, componentes, descripción & 10\% \\\hline\hline
   Total								& 100\% \\
 \end{tabular}
\end{center}

La sección ``calidad del informe'' corresponde a 10 puntos que podrán obtenerse en caso de que el informe esté escrito con correcta redacción y ortografía y esté presentado de forma correcta.

Este documento puede usarse como base para el informe, pero en el mismo deberán incluirse solamente las secciones especificadas anteriormente y nada más.

Si por motivos justificados se requiere de más tiempo para completar el informe, esta deberá ser solicitada al profesor al menos con 24 horas de anticipación.

En caso de no haber solicitado una extensión por anticipado o de haberse vencido la extensión, la máxima nota estará dada por la fórmula:

\[ M(n) = 100-\frac{1.367}{10}\cdot n^{1.367} \]

donde $M$ es la nota máxima y $n$ es la cantidad de horas de atraso en la entrega. La nota final será:

\[ F(n) = M(n)\cdot T \]

donde $T$ corresponde al porcentaje obtenido de los rubros especificados anteriormente.

\end{document}