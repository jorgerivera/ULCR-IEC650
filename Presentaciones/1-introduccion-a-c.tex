\documentclass[handout,xcolor=dvipsnames]{beamer}

\usepackage[utf8]{inputenc}
%\usepackage{default}
\usetheme[width=50pt]{ULatina}   % Bergen, Darmstadt
\usecolortheme[named=Green]{structure}
\usepackage{graphicx}
\usepackage{pgfpages}
\usepackage{tikz}
\usepackage[spanish]{babel}
\usepackage[normalem]{ulem} % to use \sout{} for strikethrough


%\setbeameroption{show notes on second screen}
%\setbeamersize{sidebar width left=50pt}


\title[C/C++]{Introducción a C/C++}
\subtitle{Programación y aplicaciones usando Arduino}
\author{Prof. Jorge Rivera~Guti\'errez}
\institute{Universidad Latina de Costa Rica\\ Ingenier\'\i a en Electr\'onica}
\logo{\includegraphics[height=45pt]{world.png}}
\date{I cuatrimestre 2014}
\newcommand{\pageframe}[1]{\frame{\begin{center}{ \Huge #1 }\end{center}}}

\begin{document}

\begin{frame}
 \maketitle
\end{frame}

\section{Variables}
\pageframe{Variables}

\begin{frame}{Tipos de Variables}
 \begin{block}{Enteros}
 \begin{tabular}{lcc}\hline
 	Tipo	& 	 Tamaño & Rango \\\hline\hline
{\tt int} &  16-bits & -32\,768 \ldots 32\,767 \\\hline
{\tt unsigned int}  & 16-bits &  0 \ldots 65\,535 \\\hline 
{\tt long}  & 32-bits & -2\,147\,483\,648 \ldots 2\,147\,483\,647 \\\hline
{\tt unsigned long}  & 32-bits &  0 \ldots 4\,294\,967\,295 \\\hline 
 \end{tabular}
 \end{block}
\end{frame}

\begin{frame}{Tipos de Variables}
 \begin{block}{Punto flotante}
 \begin{tabular}{lcc}\hline
 	Tipo	& 	 Tamaño & Rango \\\hline\hline
{\tt float} &  32-bits & -3.4028235E+38 \ldots +3.4028235E+38 \\\hline
 \end{tabular}
 \end{block}
 \begin{block}{Distribución de los bits}
 \begin{description}
 \item[1-bit] signo
 \item[8-bits] exponente
 \item[23-bits] mantisa
  \end{description}
 \end{block}
\end{frame}

\begin{frame}{Tipos de Variables}
\begin{block}{Otros tipos}
\begin{center}
 \begin{tabular}{lcc}\hline
 	Tipo	& 	 Tamaño & Rango \\\hline\hline
{\tt boolean} &  8-bits & True, False \\\hline
{\tt char} &  8-bits & 0 \ldots 255 (ASCII) \\\hline
{\tt byte} &  8-bits & 0 \ldots 255 \\\hline
{\tt void} & --- & --- \\\hline
 \end{tabular}
\end{center}
\end{block}
\end{frame}

\begin{frame}{Declaración de variables}
\begin{block}{Sin incializar}
{\tt int a;}

{\tt byte b;}

{\tt float c;}
\end{block}
\begin{block}{Con valor inicial}
{\tt int a = 100;}

{\tt byte b = 45;}

{\tt float c = 3.14;}
\end{block}
\end{frame}

\begin{frame}{Asignación de variables}
  \begin{block}{Asignaciones}
  	El símbolo = se utiliza para asignar valores. El valor obtenido al lado derecho del símbolo se introduce en la variable especificada a la izquierda.
  \end{block}
  \begin{block}
	{\tt a = b + c;}
  \end{block}
\end{frame}

\section{Funciones}

\pageframe{Funciones}


\begin{frame}{Funciones}
  \begin{block}{Funciones}
    Debe especificarse un tipo de retorno.
    
    Deben ponerse los tipos de las variables que se reciben como parámetro.
    
    El código debe estar entre llaves.
  \end{block}
\end{frame}


\end{document}
