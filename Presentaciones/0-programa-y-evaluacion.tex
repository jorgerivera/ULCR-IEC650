\documentclass[handout,xcolor=dvipsnames]{beamer}

\usepackage[utf8]{inputenc}
\usepackage{default}
\usetheme[width=50pt]{ULatina}   % Bergen, Darmstadt
\usecolortheme[named=Green]{structure}
\usepackage{graphicx}
\usepackage{pgfpages}
\usepackage{tikz}
\usepackage[spanish]{babel}

%\setbeameroption{show notes on second screen}
%\setbeamersize{sidebar width left=50pt}


\title[IEC-620]{IEC-620 Microprocesadores}
\subtitle{Programa del curso, evaluación y cronograma}
\author{Prof. Jorge Rivera~Guti\'errez}
\institute{Universidad Latina de Costa Rica\\ Ingenier\'\i a en Electr\'onica}
\logo{\includegraphics[height=45pt]{world.png}}
\date{III cuatrimestre 2013}
\newcommand{\pageframe}[1]{\frame{\begin{center}{ \Huge #1 }\end{center}}}

\begin{document}

\begin{frame}
 \maketitle
\end{frame}

\begin{frame}
 \begin{center}
  \Large Créditos: 3\\~\\
  Requisito: Sistemas Digitales II\\~\\
  Período: VII\\~\\
  Modalidad: cuatrimestral\\~\\
  Horas te\'oricas: 3\\
  Horas individuales: 6
 \end{center}
\end{frame}

\section{Descripción del curso}

\begin{frame}{Descripción del curso}
 El uso cada vez más frecuente de sistemas diseñados para realizar funciones específicas ha llevado a que el estudio de los sistemas embebidos sea un área de creciente interés para quiénes se desempeñan en carreras relacionadas con el área electrónica.

En este curso se aborda el desarrollo de soluciones dedicadas basadas en el uso de sistemas embebidos, con el fin de lograr en el estudiante la construcción de conocimientos sobre esta temática.
\end{frame}

\section{Objetivos}

\pageframe{Objetivos}

\subsection{Objetivo general}

\begin{frame}{Objetivo general}
  \begin{block}{Objetivo general}
  Desarrollar soluciones a problemas de aplicación a partir del uso de sistemas embebidos.
  \end{block}
\end{frame}

\subsection{Objetivos específicos}

\begin{frame}{Objetivos específicos}
  \begin{block}{Objetivos específicos}
    \begin{itemize}
      \item<1> Analizar aspectos generales relacionados con los sistemas embebidos.
      \item<1> Analizar aspectos generales relacionados con los controladores de interfaz periférica.
      \item<1> Analizar aspectos generales relacionados con la plataforma Arduino.
      \item<1> Desarrollar soluciones dedicadas basadas en sistemas embebidos.
    \end{itemize}
  \end{block}
\end{frame}


\section{Contenidos}

\pageframe{Contenidos}

\subsection[Principios]{Principios de Sistemas Embebidos}

\begin{frame}{Principios de Sistemas Embebidos}
  \begin{block}{}
  \begin{itemize}
    \item Arquitecturas
    \item Lenguaje Ensamblador
    \item Lenguaje C++
  \end{itemize}
  \end{block}
\end{frame}

\subsection[PICs]{PICs}

\begin{frame}{PICs}
\begin{block}{}
\begin{itemize}
 \item Características
 \item Repertorio de Instrucciones
 \item Memorias
 \item Puertos I/O
 \item Conversores A/D
 \item Comparadores
 \item Temporizadores
 \item Módulo de Comunicación
\end{itemize}
\end{block}
\end{frame}

\subsection[Arduino]{Plataforma Arduino}

\begin{frame}{Arduino}
\begin{block}{}
\begin{itemize}
 \item Estructura Interna
 \item IDE
 \item Estructura de un Sketch
 \item Variables y Constantes
 \item Funciones
 \item Bloques
 \item Estructuras de Datos
 \item Librerías Estándar
\end{itemize}
\end{block}
\end{frame}

\subsection[Desarrollo]{Desarrollo de Soluciones}

\begin{frame}{Desarrollo de Soluciones}
\begin{block}{}
\begin{itemize}
 \item Uso de LCDs
 \item Uso de la Memoria
 \item Uso de I/Os
 \item Uso de Sensores
 \item Uso de la Comunicación en Red
\end{itemize}
\end{block}
\end{frame}

\section{Metodología}

\pageframe{Metodología}

\begin{frame}{Metodología}

Este es un curso en el que se aprovechan los conocimientos y experiencias de los estudiantes para posibilitar un aprendizaje permanente y significativo, desde el punto de vista del análisis crítico, reflexivo y creativo de resolución de problemas. Con este fin se desarrollan métodos participativos y colaborativos que enriquecen la experiencia de enseñanza y aprendizaje, así como la motivación del estudiante, en un ambiente de respeto mutuo por las opiniones y en donde la ética es parte del compromiso permanente de todos los actores del proceso formativo.

\end{frame}

\section{Estrategias de aprendizaje}

\pageframe{Estrategias de aprendizaje}

\begin{frame}{Estrategias de aprendizaje}
  \begin{block}{Charlas Didácticas}
    El docente realiza la exposición de los contenidos temáticos en forma de exposición oral interactiva. A través de ella se logra la transmisión de conocimientos y se ofrece un enfoque crítico de la disciplina, con lo que se lleva a los discentes a descubrir las relaciones que existen entre los diversos conceptos.
  \end{block}
\end{frame}

\begin{frame}{Estrategias de aprendizaje}
  \begin{block}{Discusiones guiadas:}
    Permite que el docente y los estudiantes intercambien criterios acerca de un contenido temático determinado. El profesor inicia la discusión introduciendo el tema y solicita a los discentes su participación a partir de lo que conozcan del mismo. Debe participar la mayoría del grupo, de manera tal que todos escuchen y se involucren activamente en la discusión. Al final se debe generar un resumen de los tópicos abordados.  \end{block}
\end{frame}

\begin{frame}{Estrategias de aprendizaje}
  \begin{block}{Investigaciones Grupales:}
     A los estudiantes se les asignan temas que deberán desarrollar grupalmente. El profesor asume el rol de tutor en el proceso de resolución de las mismas.
  \end{block}
\end{frame}


\begin{frame}{Estrategias de aprendizaje}
  \begin{block}{Proyectos Integradores:} 
    Conllevan un desarrollo grupal de prototipos por medio de los que se genera una solución a un problema de aplicación y tienen como fin el hacer confluir diferentes temas específicos. En ellos el docente funge como tutor del proceso de generación de las soluciones.  
  \end{block}
\end{frame}

\begin{frame}{Estrategias de aprendizaje}
  \begin{block}{Resolución de Problemas:} 
    Los estudiantes abordan problemas bajo la supervisión del profesor. Permite desarrollar en ellos capacidades asociadas a la comprensión de un problema, el uso de sus conocimientos para darle una explicación, la organización de ideas y el uso de procedimientos para lograr una respuesta.
  \end{block}
\end{frame}

\begin{frame}{Estrategias de aprendizaje}
  \begin{block}{Visitas Técnicas:} 
    Son visitas de orientación técnica a sitios en los que se desarrollan actividades ligadas a objetivos específicos del curso, que permiten a los estudiantes complementar las temáticas abordadas en clase. En ellas el docente asume el rol de guía, con el fin de aprovechar al máximo la experiencia.

  \end{block}
\end{frame}


  
\section{Evaluaci\'on}

\pageframe{Evaluación}

\begin{frame}{Evaluación}
\begin{center}
\begin{tabular}{|l|r|}\hline
	2 Pruebas escritas	&	60\%\\\hline
	1 Proyecto integrador	&	30\%\\\hline
	2 Investigaciones	&	10\%\\\hline\hline
	TOTAL			&	100\%\\\hline
\end{tabular}
\end{center}
\end{frame}

\section{Observaciones}

\pageframe{Observaciones}


\begin{frame}{Presentación de informes y evaluaciones}
\begin{block}{Presentación de informes y evaluaciones}
  \begin{itemize}[<+->]
    \item Los informes y evaluaciones deberán ser realizados con una ortografía y redacción de nivel universitario.
    \item Los informes deberán ser preparados usando el formato IEEE Transactions. La descripción de este formato y las plantillas para el mismo pueden ser encontrados en \url{http://bit.ly/bN3uDr}.
    \item Los informes \textbf{deberán} ser realizados utilizando el sistema de preparación de documentos \LaTeX. 
    \item Las imágenes y los diagramas utilizados en los informes deberán ser de buena calidad y generados utilizando programas apropiados para dicha tarea.
  \end{itemize}
\end{block}
\end{frame}


\begin{frame}{Presentación de informes y evaluaciones}
\begin{block}{Presentación de informes y evaluaciones}
  \begin{itemize}[<+->]
    \item Todo documento preparado para este curso deberá ser entregado de forma impresa y su código fuente en formato digital \textbf{y en formato PDF} al correo electrónico \texttt{jm.rivera.g@gmail.com}.
    \item En los informes se descontarán hasta 10 puntos por el incumplimiento de los requerimientos de formato, redacción, ortografía y calidad del documento.
  \end{itemize}
\end{block}
\end{frame}


\begin{frame}{Asistencia}
\begin{block}{Asistencia}
  \begin{itemize}[<+->]
    \item El curso es de asistencia obligatoria. Se requiere de una asistencia al 80\% de las horas lectivas.
    \item El curso consta de 14 sesiones de teoría de 3 horas y 15 sesiones de laboratorio de 2 horas, para un total de 72 horas. Se requiere la asistencia a 58 horas de clase para aprobar el curso.
    \item Las ausencias podrán ser justificadas por iniciativa del estudiante cuando haya una razón que así lo amerite.
    \item Las ausencias a evaluaciones deberán ser tramitadas de acuerdo con lo especificado en el reglamento de la Universidad.
  \end{itemize}
\end{block}
\end{frame}

\section{Bibliografía}

\pageframe{Bibliografía}
\nocite{*}

\begin{frame}{Bibliografía}

\begin{thebibliography}{5}
 \bibitem{Texto}
  Monk, S. 
  \textit{Programming Arduino. Getting Started with Sketches}.
  Estados Unidos de América: McGraw-Hill Professional,
  2011.
 \bibitem{Cons1}
  Galeano, G.
  \textit{Programación de Sistemas Embebidos en C}.
  Mexico: Alfaomega,
  2009.
 \bibitem{Cons2}
  Torrente, O.
  \textit{Arduino. Curso Práctico de Formación}.
  McGraw Hill,
  Mexico: Alfaomega,
  2013.
 \bibitem{Cons3}
  Valdés, F. y Pallás, R.
  \textit{Microcontroladores. Fundamentos y Aplicaciones con PIC}.
  Mexico: Alfaomega,
  2007.
\end{thebibliography}

\end{frame}

\section{Cronograma}

\pageframe{Cronograma}

\begin{frame}{Cronograma}
 \begin{center}
 {\small
  \begin{tabular}{|c|c|c|}\hline
  
   Semana 	& \multicolumn{1}{c|}{Contenido Temático} & \multicolumn{1}{c|}{Actividades} \\ \hline \hline
   1 		& Principios 		&  Revisión del programa + Charla + \\ 
    		& 	 		&  Discusión + Formación de grupos \\ \hline
   2 		& Principios 		& Charla + Discusión + Expo 1 \\ \hline
   3 		& Principios 		& Charla + Discusión + Expo 1\\ \hline
   4 		& PICs		& Charla + Discusión + Expo 1 \\ \hline
   5		& PICs		& Charla + Discusión  \\ \hline
   6 		& PICs		& Charla + Discusión  \\ \hline

  
  \end{tabular}}
 \end{center}
\end{frame}

\begin{frame}{Cronograma}
 \begin{center}
 {\small
  \begin{tabular}{|c|c|c|}\hline
  
   Semana 	& \multicolumn{1}{c|}{Contenido Temático} & \multicolumn{1}{c|}{Actividades} \\ \hline \hline
   7 		& \multicolumn{2}{c|}{Prueba escrita} \\ \hline
   8 		& Arduino 		& Charla + Discusión + Expo 2 \\ \hline
   9 		& Arduino 		& Charla + Discusión + Expo 2\\ \hline
   10 		& Arduino		& Charla + Discusión + Expo 2 \\ \hline
   11		& Arduino		& Charla + Discusión  \\ \hline
   12 		& Desarrollo de soluciones & Discusión + Resolución Prob \\ \hline
   13 		& Desarrollo de soluciones & Discusión + Resolución Prob \\ \hline
   14		& Desarrollo de soluciones & Discusión + Resolución Prob \\ \hline
   15 		& \multicolumn{2}{c|}{Prueba escrita}  \\ \hline
  \end{tabular}}
 \end{center}
\end{frame}



\end{document}
