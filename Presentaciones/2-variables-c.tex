\documentclass[handout,xcolor=dvipsnames]{beamer}

\usepackage[utf8]{inputenc}
%\usepackage{default}
\usetheme[width=50pt]{ULatina}   % Bergen, Darmstadt
\usecolortheme[named=Green]{structure}
\usepackage{graphicx}
\usepackage{pgfpages}
\usepackage{tikz}
\usepackage[spanish]{babel}
\usepackage[normalem]{ulem} % to use \sout{} for strikethrough


%\setbeameroption{show notes on second screen}
%\setbeamersize{sidebar width left=50pt}


\title[C/C++]{Introducción a C/C++}
\subtitle{Variables compuestas}
\author{Prof. Jorge Rivera~Guti\'errez}
\institute{Universidad Latina de Costa Rica\\ Ingenier\'\i a en Electr\'onica}
\logo{\includegraphics[height=45pt]{world.png}}
\date{I cuatrimestre 2014}
\newcommand{\pageframe}[1]{\frame{\begin{center}{ \Huge #1 }\end{center}}}

\begin{document}

\begin{frame}
 \maketitle
\end{frame}

\section{Variables}
\pageframe{Variables Compuestas}

\begin{frame}{Tipos de Variables}
 \begin{block}{Alguos tipos de variables}
 \begin{center}
 \begin{tabular}{lcc}\hline
 	Tipo	& 	 Tamaño & Rango \\\hline\hline
{\tt int} &  16-bits & -32\,768 \ldots 32\,767 \\\hline
{\tt unsigned int}  & 16-bits &  0 \ldots 65\,535 \\\hline 
{\tt char} &  8-bits & 0 \ldots 255 (ASCII) \\\hline
{\tt byte} &  8-bits & 0 \ldots 255 \\\hline
 \end{tabular}
\end{center}
\end{block}
\end{frame}


\begin{frame}[fragile]\frametitle{Arreglos}

\begin{block}{Arreglos de variables {\tt char}}
\begin{verbatim}
char str1[15];
char str2[4] = {'l', 'a', 'b'};
char str3[4] = {'l', 'a', 'b', '\0'};
char str4[ ] = "lab";
char str5[8] = "lab";
char str6[15] = "lab";
\end{verbatim}
\end{block}

\end{frame}

\begin{frame}[fragile]\frametitle{Obteniendo un dato de un arreglo}
\begin{block}{Leer un dato}
\begin{verbatim}
char dato;
char str[12] = "laboratorio";

dato = str[0];  \\ dato 'l'
dato = str[5];  \\ dato 'a'
dato = str[11]; \\ dato '\0'
dato = str[12]; \\ ERROR! (fuera de rango)
\end{verbatim}
\end{block}
\end{frame}

\begin{frame}[fragile]\frametitle{Cambiando un dato de un arreglo}
\begin{block}{Leer un dato}
\begin{verbatim}
char dato;
char str[12] = "laboratorio";

str[0] = 'A';  \\ "Aaboratorio"
str[3] = 65;  \\ "AabAratorio"
\end{verbatim}
\end{block}
\end{frame}



\end{document}
